\documentclass[12pt,a4paper]{article}

% Encoding & language
\usepackage[utf8]{inputenc}
\usepackage[T1]{fontenc}
\usepackage[main=english]{babel}
\usepackage{caption}
\usepackage{booktabs}

% untuk kode
\usepackage{listings}

% Margin & spacing
\usepackage{geometry}
\geometry{margin=1in}
\usepackage{setspace}
\setstretch{1.2}

% Math & extras
\usepackage{enumitem}
\usepackage{amsmath,amssymb,amsthm}
\usepackage{algorithm}
\usepackage{algpseudocode}

\theoremstyle{remark}
\newtheorem*{solusi}{Solusi}

% Paket untuk gambar
\usepackage{graphicx}
\usepackage{tikz}
\usetikzlibrary{shapes.geometric, arrows, positioning}
\usetikzlibrary{calc}

% Etc
\usepackage{titling}
\usepackage{csquotes}
\usepackage{subcaption}
% Untuk hyperlink
\usepackage{hyperref}

% --- Biblatex dengan IEEE style ---
\usepackage[backend=biber,style=ieee,language=english]{biblatex}
\addbibresource{refs.bib}

\DefineBibliographyStrings{english}{%
  references   = {Daftar Pustaka},
  bibliography = {Daftar Pustaka},
  and          = {dan},
  andothers    = {dkk.},
  pages        = {hlm.},
}

% STYLING
\setlength{\parindent}{2em}

% OVERRIDE Caption
\captionsetup[figure]{labelfont=bf, name=Gambar, justification=centering}
\captionsetup[table]{labelfont=bf, name=Tabel, justification=centering}



% --- Title page ---
\title{%
  \textbf{Laporan Tugas Besar} \\
  \large Riset Operasi (RO) \\
  \large Expert Advisor pada MetaTrader 5}
  
\author{Khalilullah Al Faath \\ NIU: 566643 \\ Magister Ilmu Komputer \\ [1cm]
  \includegraphics[width=0.4\textwidth]{images/logo-ugm.png}}
\date{Universitas Gadjah Mada \\ Semester Gasal 2025/2026}

\makeatletter
\renewcommand{\maketitle}{%
  \begin{titlepage}
    \centering
    \vspace*{2cm}
    
    {\LARGE \@title \par}
    \vspace{2cm}
    
    {\large \@author \par}
    
    \vfill
    
    {\large \@date \par}
  \end{titlepage}
}
\makeatother

\begin{document}

\maketitle
\thispagestyle{empty}

% Daftar isi dan daftar gambar/tabel

% override label
\renewcommand{\contentsname}{Daftar Isi}
\renewcommand{\listfigurename}{Daftar Gambar}
\renewcommand{\listtablename}{Daftar Tabel}

% % daftar isi
% \tableofcontents

% \newpage

% % daftar gambar
% \listoffigures

% \newpage

% % daftar tabel
% \listoftables

\newpage
\setcounter{page}{1}

\section{Pendahuluan}
Pasar keuangan pasca-2021 menunjukkan karakteristik volatilitas tinggi dan fase \textit{sideways} yang berkepanjangan. Kondisi ini menuntut sistem perdagangan yang tidak hanya agresif dalam mengejar keuntungan, tetapi juga memiliki mekanisme pertahanan modal yang kuat (\textit{capital preservation}).

Penelitian ini mengembangkan \textit{Expert Advisor} (EA) bernama \textit{Golden\_UGM\_Secure\_Defender}. Berbeda dengan pendahulunya yang menggunakan risiko adaptif penuh, sistem ini mengimplementasikan model \textit{Profit Security} dua fase dan filter tren jangka panjang. Pendekatan ini dirancang untuk mengakumulasi keuntungan secara agresif pada fase awal (\textit{compounding}), kemudian mengunci modal dasar dan beroperasi dengan risiko konservatif setelah target investasi tercapai.

Selain itu, untuk mengatasi masalah \textit{false breakout} pada pasar yang datar, sistem ini mengintegrasikan filter tren berbasis \textit{Exponential Moving Average} (EMA) periode 200, yang bertindak sebagai "katup pengaman" untuk mencegah transaksi yang melawan arus utama pasar.

\section{Metodologi}

Sistem dikembangkan dengan pendekatan hibrida yang menggabungkan analisis teknikal klasik (Fraktal) dengan manajemen uang berbasis tujuan (\textit{goal-based money management}).

\subsection{Arsitektur Sistem}
Sistem beroperasi berdasarkan tiga pilar utama:
\begin{enumerate}
    \item \textbf{Modul Sinyal Terfilter:} Mengidentifikasi \textit{breakout} fraktal yang valid hanya jika searah dengan tren utama (EMA 200).
    \item \textbf{Modul Keamanan Profit (\textit{Profit Security Module}):} Memantau pencapaian target saldo untuk mengubah mode risiko dari agresif menjadi konservatif.
    \item \textbf{Modul Waktu \& Eksekusi:} Membatasi jam operasional untuk menghindari volatilitas rendah dan melebarnya \textit{spread}.
\end{enumerate}

\subsection{Strategi Masuk Pasar dengan Filter Tren}
Strategi dasar tetap menggunakan pola \textit{Fractal Breakout}, namun dengan penambahan aturan validasi tren yang ketat:
\begin{enumerate}
    \item \textbf{Identifikasi Fraktal:} Sistem mendeteksi titik \textit{High/Low} fraktal dalam periode 5 bar.
    \item \textbf{Validasi Tren (EMA 200):}
          \begin{itemize}
              \item Sinyal \textbf{BUY} hanya valid jika harga Fraktal Atas berada di \textbf{ATAS} garis EMA 200.
              \item Sinyal \textbf{SELL} hanya valid jika harga Fraktal Bawah berada di \textbf{BAWAH} garis EMA 200.
          \end{itemize}
\end{enumerate}
Mekanisme ini secara efektif mengeliminasi sinyal palsu yang sering terjadi ketika pasar bergerak melawan tren jangka panjang (\textit{counter-trend}).

\subsection{Algoritma Profit Security (Manajemen Uang)}
Kebaruan utama dalam sistem ini adalah mekanisme penguncian profit. Berbeda dengan sistem \textit{compounding} murni yang rentan terhadap penurunan drastis (\textit{drawdown}) saat saldo membesar, sistem ini membagi siklus investasi menjadi dua fase:

\subsubsection{Fase 1: Pertumbuhan Agresif (\textit{Growth Phase})}
Selama saldo akun ($B_t$) masih di bawah target yang ditentukan ($B_{target}$), volume transaksi dihitung menggunakan persentase risiko terhadap total saldo saat ini.
\begin{equation}
    B_{target} = B_{initial} \times \text{TargetMultiplier}
\end{equation}
\begin{equation}
    \text{RiskBasis} = B_t \quad \text{(Compounding Aktif)}
\end{equation}

\subsubsection{Fase 2: Pengamanan Aset (\textit{Secured Phase})}
Segera setelah saldo menyentuh atau melebihi target ($B_t \ge B_{target}$), sistem mengaktifkan mode keamanan. Perhitungan lot tidak lagi menggunakan total saldo, melainkan menggunakan basis modal aman yang tetap ($\text{SecureBaseCapital}$).
\begin{equation}
    \text{RiskBasis} = \text{SecureBaseCapital} \quad \text{(Lot Tetap/Kecil)}
\end{equation}
Mekanisme ini memastikan bahwa keuntungan berlebih (\textit{excess profit}) yang diperoleh selama fase pertumbuhan diamankan dan tidak dipertaruhkan kembali ke pasar dengan volume besar.

\section{Implementasi Sistem}

Implementasi dilakukan menggunakan MQL5 dengan fokus pada logika pemisahan fase risiko dan filter tren.

\subsection{Parameter Masukan (Input Parameters)}
Parameter dibagi menjadi grup manajemen risiko, keamanan profit, dan filter.

\begin{lstlisting}
// Money Management & Security
input double RiskPercent       = 2.0;    // Risiko per trade
input double TargetMultiplier  = 5.0;    // Target: 5x Modal Awal
input double SecureBaseCapital = 5000.0; // Basis perhitungan setelah target tercapai

// Filter Tren & Waktu
input bool   UseMaFilter       = true;   // Aktivasi Filter EMA
input int    MaPeriod          = 200;    // Periode EMA Tren Utama
input bool   UseTimeFilter     = true;   // Aktivasi Filter Waktu
\end{lstlisting}

\subsection{Logika Utama dengan Filter (OnTick)}
Fungsi `OnTick` kini memuat logika pengecekan status keamanan profit dan validasi tren sebelum eksekusi.

\begin{lstlisting}
void OnTick() {
    // 1. Cek Status Pencapaian Target
    CheckSecureStatus();

    // 2. Filter Waktu & Ambil Data EMA
    if(UseTimeFilter && !IsTradingTime()) return;
    
    double currentMa = 0;
    if(UseMaFilter) {
        CopyBuffer(maHandle, 0, 1, 1, maBuffer);
        currentMa = maBuffer[0];
    }

    // 3. Logika Entry Terfilter
    if(totalBars != bars) {
        double high = FindHigh();
        double low  = FindLow();

        // Hanya Buy jika High > EMA 200
        if(high > 0) {
            if(!UseMaFilter || (UseMaFilter && high > currentMa)) {
                ExecuteBuy(high);
            }
        }
        // Hanya Sell jika Low < EMA 200
        if(low > 0) {
            if(!UseMaFilter || (UseMaFilter && low < currentMa)) {
                ExecuteSell(low);
            }
        }
    }
}
\end{lstlisting}

\subsection{Implementasi Kalkulasi Lot Dua Fase}
Fungsi `CalcLots` dimodifikasi untuk memilih basis perhitungan risiko secara kondisional.

\begin{lstlisting}
double CalcLots(double slDist) {
    double calculationBase;
    
    // Logika Inti Profit Security:
    // Jika target tercapai, gunakan basis modal tetap ($5000)
    // Jika belum, gunakan total saldo (Compounding)
    if(isProfitSecured) {
        calculationBase = SecureBaseCapital; 
    } else {
        calculationBase = AccountInfoDouble(ACCOUNT_BALANCE); 
    }

    double riskMoney = calculationBase * (RiskPercent / 100.0);
    // ... (dilanjutkan dengan konversi ke lot dan normalisasi)
    return lot;
}
\end{lstlisting}

\subsection{Fungsi Pemicu Keamanan}
Fungsi ini memantau saldo secara real-time dan mengunci status sistem secara permanen jika target tercapai.

\begin{lstlisting}
void CheckSecureStatus() {
    if(isProfitSecured) return; // Jika sudah aman, lewati

    double currentBalance = AccountInfoDouble(ACCOUNT_BALANCE);
    double targetAmount = initialBalance * TargetMultiplier;

    if(currentBalance >= targetAmount) {
        isProfitSecured = true;
        DeletePendingOrders(); // Reset order lama dengan lot baru
        Print("TARGET TERCAPAI! Mode Pengamanan Diaktifkan.");
    }
}
\end{lstlisting}
\section{Hasil dan Pembahasan}

Pengujian kinerja strategi (\textit{backtest}) dilakukan menggunakan data historis dengan kualitas pemodelan \textit{real ticks} (49\%). Simulasi dijalankan pada rentang waktu jangka panjang, mencakup berbagai kondisi pasar mulai dari Januari 2019 hingga Oktober 2025.

\subsection{Parameter Pengujian}
Pengujian dilakukan dengan modal awal sebesar USD 10.000. Parameter utama sistem yang digunakan dalam pengujian ini dirancang untuk menguji ketahanan strategi \textit{Mean Reversion} (RSI) yang dikombinasikan dengan mekanisme \textit{Profit Security}.

\begin{table}[H]
    \centering
    \caption{Parameter Utama Expert Advisor}
    \label{tab:param}
    \begin{tabular}{ll}
        \toprule
        Parameter     & Nilai         \\
        \midrule
        Risk Percent  & 2.0\%         \\
        Strategy Type & RSI Reversion \\
        Timeframe     & H1            \\
        Take Profit   & 200 poin      \\
        Stop Loss     & 200 poin      \\
        Trailing Stop & 5 poin        \\
        \bottomrule
    \end{tabular}
\end{table}

\subsection{Statistik Kinerja Utama}
Ringkasan hasil pengujian berdasarkan data terbaru disajikan pada Tabel~\ref{tab:hasil}. Data menunjukkan bahwa meskipun strategi mampu menghasilkan keuntungan bersih yang signifikan, terdapat tingkat risiko (\textit{drawdown}) yang perlu menjadi perhatian.

\begin{table}[H]
    \centering
    \caption{Ringkasan Statistik Hasil Backtest (2019-2025)}
    \label{tab:hasil}
    \begin{tabular}{lrr}
        \toprule
        \textbf{Metrik Kinerja}   & \textbf{Nilai}          & \textbf{Keterangan}           \\
        \midrule
        Modal Awal                & USD 10,000.00           & -                             \\
        \textbf{Total Net Profit} & \textbf{USD 177,853.16} & Pertumbuhan $\approx$ 1,778\% \\
        Gross Profit              & USD 1,462,354.97        & -                             \\
        Gross Loss                & USD -1,284,501.81       & -                             \\
        \midrule
        \textbf{Profit Factor}    & \textbf{1.14}           & Marjinal (Profit $>$ Loss)    \\
        Recovery Factor           & 1.27                    & Pemulihan Moderat             \\
        Sharpe Ratio              & 33.42                   & -                             \\
        Expected Payoff           & 330.58                  & Rata-rata profit/trade        \\
        \midrule
        Total Trades              & 538                     & -                             \\
        \textbf{Win Rate (Total)} & \textbf{42.19\%}        & (227 Menang / 311 Kalah)      \\
        Short Trades Won          & 40.64\%                 & -                             \\
        Long Trades Won           & 43.02\%                 & -                             \\
        \bottomrule
    \end{tabular}
\end{table}

Bukti autentik hasil \textit{backtest} dari platform MetaTrader 5 ditampilkan pada Gambar~\ref{fig:hasil-backtest}. Data menunjukkan bahwa EA menghasilkan \textit{Total Net Profit} sebesar USD 177,853.16. Perlu dicatat bahwa \textit{Win Rate} berada di angka 42.19\%, yang wajar untuk strategi berbasis rasio risiko:hasil (\textit{Risk:Reward}) yang seimbang atau positif.

\begin{figure}[H]
    \centering
    % Masukkan nama file screenshot tabel statistik yang baru (image_46f5c4.png)
    \includegraphics[width=1\textwidth]{images/statistik.png}
    \caption{Ringkasan Statistik Hasil Backtest dari MetaTrader 5}
    \label{fig:hasil-backtest}
\end{figure}

\subsection{Analisis Visual Pertumbuhan Aset}
Visualisasi pertumbuhan aset ditampilkan pada Gambar~\ref{fig:equity_curve}. Grafik menunjukkan dua fase yang berbeda:
\begin{enumerate}
    \item \textbf{Fase Pertumbuhan (2019 - Pertengahan 2021):} Ekuitas mengalami kenaikan tajam, menunjukkan efektivitas strategi pada kondisi pasar tertentu (kemungkinan tren kuat atau volatilitas tinggi yang sesuai dengan logika EA).
    \item \textbf{Fase Penurunan/Konsolidasi (Akhir 2021 - 2025):} Terjadi penurunan (\textit{drawdown}) yang berkepanjangan secara perlahan. Hal ini terlihat dari kurva ekuitas yang melandai turun.
\end{enumerate}

\begin{figure}[H]
    \centering
    % Masukkan nama file screenshot grafik balance/equity yang baru (image_47ccb7.png)
    \includegraphics[width=1.0\textwidth]{images/balance-equity.png}
    \caption{Kurva Pertumbuhan Ekuitas (2019-2025)}
    \label{fig:equity_curve}
\end{figure}

Fenomena ini mengonfirmasi bahwa strategi sangat sensitif terhadap rezim pasar. Mekanisme \textit{Profit Security} terlihat bekerja dengan menjaga saldo agar tidak jatuh di bawah modal awal meskipun terjadi penurunan performa strategi di tahun-tahun terakhir.

\subsection{Analisis Profitabilitas dan Risiko}
Nilai \textbf{Profit Factor sebesar 1.14} menunjukkan bahwa strategi ini masih menguntungkan secara matematis, namun margin keamanannya tipis. Untuk setiap USD 1 kerugian, sistem menghasilkan USD 1.14 keuntungan.

Tingkat risiko tercatat cukup tinggi dengan \textbf{Maximal Drawdown sebesar 56.30\%}. Angka ini mengindikasikan bahwa pada satu titik, nilai akun turun lebih dari separuh dari titik tertingginya. Meskipun sistem berhasil memulihkan diri dan tetap mencetak profit total yang besar, volatilitas ini menunjukkan perlunya manajemen risiko tambahan (seperti filter tren MA 200 yang telah diusulkan) untuk menahan penurunan di fase pasar yang tidak kondusif (2022-2025).

\subsection{Kesimpulan Hasil}
Secara keseluruhan, sistem terbukti mampu bertahan dan menghasilkan keuntungan jangka panjang yang substansial (USD 177k dari modal USD 10k). Namun, penurunan performa pasca-2021 menyoroti pentingnya adaptasi strategi, seperti penggunaan filter tren atau mekanisme \textit{switching} strategi saat kondisi pasar berubah dari \textit{trending} menjadi \textit{sideways} atau sebaliknya.

\section{Kesimpulan dan Saran}

\subsection{Kesimpulan}
Berdasarkan pengembangan dan pengujian sistem perdagangan otomatis \textit{Golden\_UGM\_Secure\_Defender} selama periode 2019--2025, dapat ditarik beberapa kesimpulan utama:

\begin{enumerate}
    \item \textbf{Profitabilitas Jangka Panjang:} Sistem terbukti mampu menghasilkan keuntungan substansial secara akumulatif. Dengan modal awal USD 10.000, sistem membukukan \textit{Total Net Profit} sebesar USD 177,853.16 (ROI $>$ 1.700\%). Hal ini memvalidasi bahwa strategi \textit{fractal breakout} yang dikombinasikan dengan manajemen modal agresif sangat efektif pada kondisi pasar yang sedang tren (\textit{trending market}), seperti yang terjadi pada periode 2019 hingga pertengahan 2021.

    \item \textbf{Efektivitas Mekanisme \textit{Profit Security}:} Fitur keamanan profit terbukti krusial dalam menjaga kelangsungan akun. Pada fase pasca-2021, ketika kondisi pasar berubah menjadi tidak kondusif (\textit{choppy/sideways}), kurva ekuitas menunjukkan penurunan yang melandai, bukan kejatuhan vertikal. Hal ini membuktikan bahwa mekanisme penguncian lot (berbasis modal dasar USD 5.000 setelah target tercapai) berhasil membatasi eksposur risiko dan mencegah \textit{Margin Call} saat strategi mengalami stagnasi.

    \item \textbf{Sensitivitas Terhadap Rezim Pasar:} Penurunan kinerja yang terlihat pada kurva ekuitas mulai akhir 2021 hingga 2025 mengindikasikan bahwa strategi ini memiliki ketergantungan tinggi pada volatilitas terarah. Nilai \textit{Profit Factor} sebesar 1.14 dan \textit{Drawdown} maksimal sebesar 56.30\% menunjukkan bahwa sistem mengalami kesulitan dalam mempertahankan tepi statistik (\textit{statistical edge}) pada pasar yang datar, meskipun telah dilengkapi dengan filter tren EMA 200.
\end{enumerate}

\subsection{Saran}
Untuk pengembangan selanjutnya guna mengatasi penurunan kinerja pada fase pasar \textit{sideways}, disarankan beberapa perbaikan berikut:

\begin{enumerate}
    \item \textbf{Implementasi Deteksi Rezim Pasar (\textit{Regime Switching}):} Sistem perlu dilengkapi dengan algoritma untuk mendeteksi perubahan kondisi pasar dari \textit{trending} ke \textit{sideways} (misalnya menggunakan indikator ADX atau Volatilitas). Sistem sebaiknya secara otomatis menonaktifkan strategi \textit{Breakout} dan beralih ke strategi \textit{Mean Reversion} (seperti RSI) saat tren melemah.

    \item \textbf{Optimasi Filter Waktu:} Mengingat banyaknya sinyal palsu yang terjadi pada sesi perdagangan dengan likuiditas rendah, pembatasan jam operasional yang lebih ketat (hanya pada Sesi London dan New York) dapat diterapkan untuk meningkatkan kualitas sinyal masuk.

    \item \textbf{Diversifikasi Aset:} Mengingat risiko \textit{drawdown} yang cukup tinggi (56\%), disarankan untuk tidak mengaplikasikan sistem ini hanya pada satu pasangan mata uang (EURUSD), melainkan pada portofolio aset yang tidak berkorelasi untuk membagi risiko.
\end{enumerate}

\newpage

% Daftar pustaka
\printbibliography

\end{document}
