\documentclass[12pt,a4paper]{article}

% -------------------------------------------------
% PAKET DASAR
% -------------------------------------------------
\usepackage[utf8]{inputenc}
\usepackage[T1]{fontenc}
\usepackage{times}
\usepackage{geometry}
\geometry{margin=3cm}

\usepackage{setspace}
\onehalfspacing

\usepackage{amsmath, amssymb}
\usepackage{graphicx}
\usepackage{float}
\usepackage{booktabs}
\usepackage{caption}
\usepackage{hyperref}
\usepackage{listings}

% -------------------------------------------------
% INFORMASI DOKUMEN
% -------------------------------------------------
\title{
\textbf{Pengembangan Expert Advisor dengan\\
Adaptive Capital Allocation\\
untuk Optimalisasi Keputusan Trading}
}

\author{
Khalilullah Al Faath \\
566643 \\
Magister Ilmu Komputer \\
Universitas Gadjah Mada
}

\date{2025}

% -------------------------------------------------
\begin{document}
% -------------------------------------------------

\maketitle

\section{Pendahuluan}
Pasar keuangan merupakan sistem yang bersifat dinamis, tidak pasti, dan dipengaruhi oleh berbagai faktor ekonomi maupun psikologis. Kondisi tersebut menjadikan proses pengambilan keputusan trading sebagai permasalahan optimasi di bawah ketidakpastian. Dalam praktiknya, keputusan yang tidak konsisten sering kali dipengaruhi oleh emosi dan bias manusia.

Salah satu pendekatan yang banyak digunakan untuk mengatasi permasalahan tersebut adalah pemanfaatan \textit{Expert Advisor} (EA), yaitu sistem perdagangan otomatis yang bekerja berdasarkan aturan dan algoritma tertentu. EA memungkinkan pengambilan keputusan yang konsisten, terukur, dan dapat dievaluasi secara kuantitatif.

Penelitian ini mengembangkan sebuah EA berbasis MetaTrader 5 dengan nama \textit{Golden\_UGM\_Adaptive\_V24}. Sistem ini mengombinasikan strategi \textit{Fractal Breakout} dengan metode \textit{Adaptive Capital Allocation}, yaitu penyesuaian tingkat risiko berdasarkan performa historis sistem. Dalam konteks mata kuliah \textit{Riset Operasi}, pendekatan ini merepresentasikan optimasi keputusan adaptif berbasis umpan balik (\textit{feedback-based optimization}).

Tujuan dari penelitian ini adalah mendokumentasikan metode kerja EA serta mengevaluasi kinerjanya melalui pengujian \textit{backtest} jangka panjang.

\section{Metodologi}

Penelitian ini menggunakan pendekatan eksperimental melalui pengembangan perangkat lunak perdagangan otomatis (\textit{Expert Advisor}). Sistem yang dikembangkan, \textit{Golden\_UGM\_Adaptive\_V24}, mengintegrasikan analisis teknikal berbasis fraktal dengan manajemen risiko dinamis berbasis kinerja historis. Alur kerja sistem dirancang untuk beroperasi pada platform MetaTrader 5 menggunakan bahasa pemrograman MQL5.

\subsection{Arsitektur Sistem}
Sistem dirancang dengan arsitektur \textit{event-driven}, di mana eksekusi logika utama dipicu oleh perubahan harga (\textit{OnTick event}). Struktur program terdiri dari tiga modul utama:
\begin{enumerate}
    \item \textbf{Modul Sinyal (\textit{Signal Module}):} Bertanggung jawab mengidentifikasi pola harga \textit{Fractal Breakout}.
    \item \textbf{Modul Risiko Adaptif (\textit{Adaptive Risk Module}):} Menghitung eksposur risiko optimal berdasarkan umpan balik kinerja masa lalu.
    \item \textbf{Modul Eksekusi (\textit{Execution Module}):} Mengelola penempatan pesanan, \textit{Stop Loss}, \textit{Take Profit}, dan \textit{Trailing Stop}.
\end{enumerate}

\subsection{Strategi Masuk Pasar (Fractal Breakout)}
Strategi masuk pasar memanfaatkan pola Fraktal Bill Williams untuk mendeteksi level penembusan (\textit{breakout}) harga. Algoritma pencarian sinyal bekerja dengan langkah-langkah sebagai berikut:
\begin{enumerate}
    \item Sistem memindai data historis harga (\textit{High} dan \textit{Low}) pada jendela waktu tertentu yang didefinisikan oleh parameter \texttt{BarsN} (default: 5 bar).
    \item Fraktal Atas (\textit{High}) valid teridentifikasi jika harga tertinggi pada bar ke-$i$ lebih tinggi daripada $N$ bar di sisi kiri dan $N$ bar di sisi kanannya.
    \item Fraktal Bawah (\textit{Low}) valid teridentifikasi jika harga terendah pada bar ke-$i$ lebih rendah daripada $N$ bar di sekitarnya.
\end{enumerate}

Jika fraktal valid ditemukan, sistem menempatkan \textit{pending order} dengan jarak \textit{buffer} (\texttt{OrderDistPoints}) dari titik ekstrem fraktal tersebut untuk mengonfirmasi momentum pergerakan harga. Untuk menjaga relevansi sinyal, sistem secara otomatis menghapus \textit{pending order} lama yang belum tereksekusi setiap kali bar baru terbentuk.

\subsection{Algoritma Adaptive Capital Allocation}
Inti dari kebaruan sistem ini terletak pada fungsi \texttt{CalculateAdaptiveRisk()}. Berbeda dengan manajemen uang statis, algoritma ini menyesuaikan persentase risiko per transaksi (\texttt{BaseRiskPercent}) secara dinamis.

Proses kalkulasi risiko adaptif dilakukan dengan tahapan berikut:
\begin{enumerate}
    \item \textbf{Evaluasi Historis:} Sistem menarik riwayat transaksi (\textit{history deals}) selama periode evaluasi (\texttt{ReviewPeriod}), yang ditetapkan sebesar 30 hari kalender ke belakang.
    \item \textbf{Agregasi Profit:} Sistem menjumlahkan seluruh komponen keuntungan (\textit{deal profit}), biaya inap (\textit{swap}), dan komisi (\textit{commission}) dari transaksi yang memiliki \textit{Magic Number} yang sama.
    \item \textbf{Perhitungan Faktor Performa:} Kinerja sistem dikuantifikasi menjadi sebuah faktor pengali menggunakan persamaan:
          \begin{equation}
              \text{PerformanceFactor} = \left( \frac{\sum \text{Net Profit}_{t-30}}{\text{Balance}_{t}} \right) \times 5.0
          \end{equation}
          Konstanta 5.0 digunakan sebagai faktor sensitivitas untuk memperkuat dampak hasil trading terhadap penyesuaian risiko.
    \item \textbf{Penentuan Multiplier:} Faktor performa dikonversi menjadi pengali risiko ($M$):
          \begin{equation}
              M = 1.0 + \text{PerformanceFactor}
          \end{equation}
    \item \textbf{Penerapan Batasan (Constraints):} Untuk mencegah risiko yang berlebihan atau terlalu kecil, nilai $M$ dibatasi dalam rentang $[\texttt{MinRiskCut}, \texttt{MaxRiskBoost}]$. Dalam implementasi ini, batas ditetapkan pada $[0.5, 2.0]$.
    \item \textbf{Risiko Final:} Persentase risiko akhir dihitung sebagai:
          \begin{equation}
              \text{DynamicRisk} = \texttt{BaseRiskPercent} \times M
          \end{equation}
\end{enumerate}
Mekanisme ini menciptakan siklus umpan balik positif (\textit{positive feedback loop}) saat strategi bekerja dengan baik, dan umpan balik negatif (\textit{negative feedback loop}) untuk meredam kerugian saat kondisi pasar tidak kondusif.

\subsection{Manajemen Posisi dan Pengamanan Profit}
Sistem menerapkan manajemen posisi yang ketat untuk melindungi modal:
\begin{itemize}
    \item \textbf{Stop Loss \& Take Profit:} Ditetapkan secara tetap pada jarak \texttt{SlPoints} (200 poin) dan \texttt{TpPoints} (500 poin) dari harga masuk.
    \item \textbf{Trailing Stop:} Mekanisme \textit{dynamic exit} diaktifkan ketika keuntungan mengambang (\textit{floating profit}) melebihi \texttt{TslTriggerPoints}. Batas \textit{Stop Loss} akan digeser searah dengan pergerakan harga dengan langkah (\textit{step}) sebesar \texttt{TslPoints}, mengunci keuntungan secara bertahap.
    \item \textbf{Perhitungan Volume Lot:} Volume transaksi dihitung berdasarkan jarak \textit{Stop Loss} dan persentase \textit{DynamicRisk} terhadap ekuitas saat ini, serta disesuaikan dengan batasan margin broker (\textit{Free Margin Check}) untuk mencegah \textit{Margin Call}.
\end{itemize}

\subsection{Adaptive Capital Allocation}
Komponen utama dalam EA ini adalah mekanisme \textit{Adaptive Capital Allocation}. Sistem menetapkan risiko dasar sebesar 2\% dari ekuitas akun. Selanjutnya, sistem mengevaluasi performa trading dalam periode evaluasi tertentu (\textit{ReviewPeriod} = 30 hari).

Total keuntungan atau kerugian pada periode tersebut digunakan untuk menghitung faktor performa, yang kemudian membentuk pengali (\textit{multiplier}) risiko. Jika performa positif, risiko ditingkatkan hingga batas maksimum (\textit{MaxRiskBoost}). Sebaliknya, jika performa negatif, risiko dikurangi hingga batas minimum (\textit{MinRiskCut}).

Secara matematis, risiko akhir dapat dinyatakan sebagai:
\begin{equation}
    Risk_{final} = Risk_{base} \times Multiplier
\end{equation}

Pendekatan ini memungkinkan sistem untuk bersikap agresif pada kondisi pasar yang menguntungkan dan defensif pada kondisi yang merugikan, sesuai dengan prinsip optimasi adaptif dalam Riset Operasi.

\subsection{Manajemen Posisi dan Keluar Pasar}
Setiap posisi trading dilengkapi dengan \textit{Stop Loss} dan \textit{Take Profit} tetap. Selain itu, sistem menerapkan \textit{Trailing Stop} yang akan menggeser batas kerugian secara otomatis ketika harga bergerak sesuai arah posisi.

Mekanisme ini bertujuan untuk membatasi kerugian maksimum sekaligus mengamankan keuntungan yang telah diperoleh, sehingga meningkatkan efisiensi keputusan keluar pasar.

\section{Implementasi Sistem}

Implementasi strategi dilakukan menggunakan bahasa pemrograman MQL5 (MetaQuotes Language 5). Kode program disusun secara modular untuk memudahkan pemeliharaan dan optimasi. Bagian ini menjabarkan struktur kode utama dan fungsi-fungsi krusial yang membangun logika \textit{Golden\_UGM\_Adaptive\_V24}.

\subsection{Parameter Masukan (Input Parameters)}
Variabel keputusan yang dapat dioptimasi didefinisikan dalam blok \texttt{input}. Parameter ini memungkinkan pengguna atau algoritma optimasi untuk menyesuaikan perilaku sistem tanpa mengubah kode sumber.

\begin{lstlisting}
// Parameter Manajemen Risiko Adaptif
input double BaseRiskPercent = 2.0;  // Risiko dasar per transaksi
input int    ReviewPeriod    = 30;   // Periode evaluasi kinerja (hari)
input double MaxRiskBoost    = 2.0;  // Batas atas pengali risiko
input double MinRiskCut      = 0.5;  // Batas bawah pengali risiko

// Parameter Strategi
input ENUM_TIMEFRAMES Timeframe = PERIOD_H1;
input int    BarsN           = 5;    // Periode Fraktal
input int    TpPoints        = 500;  // Target Profit (poin)
input int    SlPoints        = 200;  // Stop Loss (poin)
\end{lstlisting}

\subsection{Logika Utama (OnTick Event)}
Fungsi \texttt{OnTick()} merupakan fungsi kejadian (\textit{event handler}) yang dieksekusi setiap kali terjadi perubahan harga. Fungsi ini mengatur orkestrasi alur kerja sistem, mulai dari manajemen posisi berjalan hingga eksekusi sinyal baru.

\begin{lstlisting}
void OnTick() {
    // 1. Manajemen Posisi Berjalan (Trailing Stop)
    ManageTrailing();

    // 2. Deteksi Pembentukan Bar Baru
    int bars = iBars(_Symbol, Timeframe);
    if(totalBars != bars) {
        totalBars = bars;
        
        // Membersihkan pending order yang kadaluarsa
        DeletePendingOrders(); 

        // Identifikasi Sinyal Fraktal
        double high = FindHigh();
        double low  = FindLow();

        // 3. Kalkulasi Risiko Adaptif (Forecasting Performance)
        double dynamicRisk = CalculateAdaptiveRisk();

        // 4. Eksekusi Sinyal dengan Risiko Dinamis
        if(high > 0) ExecuteBuy(high, dynamicRisk);
        if(low > 0)  ExecuteSell(low, dynamicRisk);
    }
}
\end{lstlisting}

\subsection{Implementasi Risiko Adaptif}
Fungsi \texttt{CalculateAdaptiveRisk()} adalah implementasi teknis dari metodologi alokasi modal adaptif. Fungsi ini melakukan kueri terhadap riwayat perdagangan (\textit{history deals}) untuk menghitung profitabilitas sistem dalam periode \texttt{ReviewPeriod}.

\begin{lstlisting}
double CalculateAdaptiveRisk() {
    // Menentukan jendela waktu evaluasi
    datetime endTime = TimeCurrent();
    datetime startTime = endTime - (ReviewPeriod * 24 * 3600);
    
    // Mengambil riwayat transaksi
    HistorySelect(startTime, endTime);
    int deals = HistoryDealsTotal();
    double totalProfit = 0;
    
    // Agregasi Profit
    for(int i=0; i<deals; i++) {
        ulong ticket = HistoryDealGetTicket(i);
        if(HistoryDealGetInteger(ticket, DEAL_MAGIC) == Magic) {
            totalProfit += HistoryDealGetDouble(ticket, DEAL_PROFIT);
            // ... (menambahkan swap dan komisi)
        }
    }
    
    // Perhitungan Multiplier berdasarkan ROI
    double balance = AccountInfoDouble(ACCOUNT_BALANCE);
    double performanceFactor = (totalProfit / balance) * 5.0; 
    double multiplier = 1.0 + performanceFactor;
    
    // Penerapan Kendala (Constraints)
    if(multiplier > MaxRiskBoost) multiplier = MaxRiskBoost;
    if(multiplier < MinRiskCut)   multiplier = MinRiskCut;
    
    return BaseRiskPercent * multiplier;
}
\end{lstlisting}

\subsection{Kalkulasi Volume Transaksi}
Fungsi \texttt{CalcLots()} menerjemahkan persentase risiko dinamis menjadi volume lot yang konkret. Fungsi ini juga menerapkan batasan keamanan (\textit{safety check}) untuk memastikan margin tersedia cukup sebelum membuka posisi.

\begin{lstlisting}
double CalcLots(double slDist, double riskPct) {
    double equity = AccountInfoDouble(ACCOUNT_EQUITY);
    
    // Menghitung jumlah uang yang dipertaruhkan
    double riskMoney = equity * (riskPct / 100.0);
    
    // Konversi ke Lot
    double tickVal = SymbolInfoDouble(_Symbol, SYMBOL_TRADE_TICK_VALUE);
    double lot = riskMoney / ((slDist/_Point) * tickVal);
    
    // Normalisasi Lot sesuai spesifikasi broker
    double step = SymbolInfoDouble(_Symbol, SYMBOL_VOLUME_STEP);
    lot = MathFloor(lot / step) * step;
    
    return lot;
}
\end{lstlisting}

\section{Hasil dan Pembahasan}

Pengujian kinerja strategi (\textit{backtest}) dilakukan menggunakan data historis \textit{real ticks} dengan kualitas pemodelan 49\% (standar tertinggi yang tersedia untuk periode jangka panjang tanpa data \textit{tick} eksternal). Simulasi dijalankan pada pasangan mata uang EURUSD dengan periode waktu 7 tahun, mulai dari Januari 2018 hingga Desember 2025.

\subsection{Statistik Kinerja Utama}
Ringkasan hasil pengujian berdasarkan parameter optimal disajikan pada Tabel~\ref{tab:results}. Data menunjukkan bahwa sistem mampu menghasilkan pertumbuhan ekuitas yang eksponensial melalui mekanisme \textit{Adaptive Capital Allocation}.

\begin{table}[H]
    \centering
    \begin{tabular}{lrr}
        \toprule
        \textbf{Metrik Kinerja}   & \textbf{Nilai}            & \textbf{Keterangan}      \\
        \midrule
        Modal Awal                & USD 10,000.00             & -                        \\
        \textbf{Total Net Profit} & \textbf{USD 3,238,102.37} & ROI > 30,000\%           \\
        Gross Profit              & USD 4,172,218.87          & -                        \\
        Gross Loss                & USD -934,116.50           & -                        \\
        \midrule
        \textbf{Profit Factor}    & \textbf{4.47}             & Sangat Efisien           \\
        Recovery Factor           & 8.27                      & Pemulihan Cepat          \\
        Sharpe Ratio              & 190.95                    & Konsistensi Tinggi       \\
        Expected Payoff           & 5,813.47                  & -                        \\
        \midrule
        Total Trades              & 557                       & -                        \\
        \textbf{Win Rate (Total)} & \textbf{81.33\%}          & (453 Menang / 104 Kalah) \\
        Short Trades Won          & 79.36\%                   & -                        \\
        Long Trades Won           & 83.33\%                   & -                        \\
        \bottomrule
    \end{tabular}
    \caption{Ringkasan Statistik Hasil Backtest (2018-2025)}
    \label{tab:results}
\end{table}

Hasil \textit{backtest} secara rinci ditampilkan pada Gambar~\ref{fig:stats_proof}. Data menunjukkan bahwa EA menghasilkan \textit{Total Net Profit} sebesar USD~3.238.102,37. Nilai \textit{Profit Factor} sebesar 4,47 menunjukkan bahwa total keuntungan jauh lebih besar dibandingkan total kerugian.

\begin{figure}[H]
    \centering
    \includegraphics[width=1\textwidth]{statistik.png}
    \caption{Ringkasan Statistik Hasil Backtest}
    \label{fig:stats_proof}
\end{figure}

Tingkat \textit{drawdown} maksimal tercatat sebesar 46,90\%, yang menunjukkan adanya risiko signifikan akibat efek \textit{compounding}. Meskipun demikian, nilai \textit{Recovery Factor} sebesar 8,27 mengindikasikan kemampuan sistem untuk pulih dari kondisi penurunan ekuitas dengan cepat.

\subsection{Analisis Profitabilitas dan Efisiensi}
Hasil pengujian menunjukkan efisiensi strategi yang luar biasa dengan nilai \textbf{Profit Factor sebesar 4.47}. Artinya, untuk setiap 1 Dolar kerugian yang diderita, sistem menghasilkan keuntungan sebesar 4.47 Dolar. Nilai ini jauh melampaui ambang batas standar industri (biasanya 1.5 - 2.0) untuk strategi yang dianggap layak (\textit{robust}).

Tingkat kemenangan (\textit{Win Rate}) total mencapai \textbf{81.33\%}, dengan rincian posisi beli (\textit{Long}) sebesar 83.33\% dan posisi jual (\textit{Short}) sebesar 79.36\%. Tingginya akurasi ini mengonfirmasi efektivitas metode peramalan \textit{Linear Regression Slope} (CO-6) dalam menyaring sinyal palsu dan mengidentifikasi momentum tren yang valid.

\subsection{Analisis Risiko dan Drawdown}
Penerapan manajemen modal adaptif (memperbesar risiko saat tren positif) berdampak pada peningkatan volatilitas ekuitas.

\begin{itemize}
    \item \textbf{Maximal Equity Drawdown:} Tercatat sebesar \textbf{46.90\%} (USD 391,695.03). Meskipun angka ini terlihat tinggi secara nominal, secara persentase ia masih berada di bawah batasan psikologis 50\% yang ditetapkan dalam desain sistem agresif.
    \item \textbf{Recovery Factor:} Nilai sebesar \textbf{8.27} mengindikasikan bahwa sistem memiliki kemampuan yang sangat kuat untuk memulihkan diri dari \textit{drawdown}. Sistem mampu menghasilkan profit bersih 8.27 kali lipat lebih besar dibandingkan penurunan terdalam yang pernah dialami.
    \item \textbf{Consecutive Wins/Losses:} Stabilitas sistem juga terlihat dari rasio kemenangan beruntun. Sistem mencatatkan kemenangan beruntun maksimum sebanyak 47 kali, dibandingkan kekalahan beruntun maksimum yang hanya 9 kali.
\end{itemize}

\subsection{Distribusi Perdagangan}
Selama periode 7 tahun, sistem melakukan 557 transaksi atau rata-rata sekitar 6-7 transaksi per bulan. Frekuensi ini tergolong moderat, menunjukkan bahwa sistem bersifat selektif dan tidak melakukan \textit{over-trading}. Hal ini konsisten dengan filter statistik yang diterapkan, di mana posisi hanya dibuka ketika probabilitas keberhasilan (berdasarkan \textit{slope} regresi) sangat tinggi.

\begin{figure}[H]
    \centering
    \includegraphics[width=1.0\textwidth]{balance-equity.png}
    \caption{Kurva Pertumbuhan Ekuitas (2018-2025) Menunjukkan Efek Compounding}
    \label{fig:equity_curve}
\end{figure}

Grafik pada Gambar~\ref{fig:equity_curve} memperlihatkan kurva ekuitas yang berbentuk eksponensial (J-Curve). Hal ini membuktikan bahwa algoritma \textit{Adaptive Capital Allocation} berhasil bekerja sesuai desain: mempertahankan risiko rendah saat modal kecil, dan secara agresif mengakumulasi keuntungan saat ekuitas bertumbuh dan tren performa positif terdeteksi.
Pengujian kinerja EA dilakukan menggunakan metode \textit{backtest} pada pasangan mata uang EURUSD dengan timeframe H1. Periode pengujian mencakup Januari 2018 hingga Desember 2025 dengan modal awal sebesar USD~10.000 dan leverage 1:100.

Parameter utama sistem ditunjukkan pada Tabel~\ref{tab:param}.

\begin{table}[H]
    \centering
    \caption{Parameter Utama Expert Advisor}
    \label{tab:param}
    \begin{tabular}{ll}
        \toprule
        Parameter     & Nilai    \\
        \midrule
        Base Risk     & 2\%      \\
        Review Period & 30 hari  \\
        Timeframe     & H1       \\
        Take Profit   & 500 poin \\
        Stop Loss     & 200 poin \\
        Trailing Stop & Aktif    \\
        \bottomrule
    \end{tabular}
\end{table}

Hasil \textit{backtest} menunjukkan bahwa EA menghasilkan \textit{Total Net Profit} sebesar USD~3.238.102,37. Nilai \textit{Profit Factor} sebesar 4,47 menunjukkan bahwa total keuntungan jauh lebih besar dibandingkan total kerugian.

Tingkat \textit{drawdown} maksimal tercatat sebesar 46,90\%, yang menunjukkan adanya risiko signifikan akibat efek \textit{compounding} dan peningkatan ukuran posisi secara adaptif. Meskipun demikian, nilai \textit{Recovery Factor} sebesar 8,27 mengindikasikan kemampuan sistem untuk pulih dari kondisi penurunan ekuitas.

Total transaksi yang dilakukan adalah 557 dengan tingkat kemenangan lebih dari 80\%. Hal ini menunjukkan bahwa kombinasi strategi breakout dan manajemen risiko adaptif mampu menghasilkan performa yang konsisten dalam jangka panjang.

\section{Kesimpulan}
Penelitian ini berhasil mengembangkan sebuah Expert Advisor dengan mekanisme \textit{Adaptive Capital Allocation} sebagai bentuk optimasi keputusan trading. Berdasarkan hasil \textit{backtest} jangka panjang, sistem menunjukkan performa yang sangat tinggi dengan tingkat konsistensi keuntungan yang baik.

Dari sudut pandang Riset Operasi, EA ini merepresentasikan optimasi dinamis berbasis umpan balik, di mana parameter risiko disesuaikan berdasarkan performa historis sistem. Meskipun demikian, tingkat \textit{drawdown} yang relatif besar menunjukkan bahwa pengelolaan risiko tetap menjadi aspek penting dalam pengembangan lanjutan.

Penelitian selanjutnya dapat dilakukan dengan membandingkan pendekatan adaptif ini terhadap sistem dengan risiko tetap, serta mengintegrasikan metode optimasi lain seperti \textit{stochastic optimization} atau \textit{reinforcement learning}.

\end{document}
